\documentclass{article}
\usepackage[utf8]{inputenc} % UTF-8
\usepackage{amsmath}
\usepackage{amssymb}
\usepackage{algpseudocode} % Algoritmos
\usepackage[a4paper, total={6in, 8in}]{geometry}
\usepackage{graphicx} % Figuras e imagenes
\usepackage{multirow} % Combinacion de celdas en tablas
\usepackage{algorithmicx}
\usepackage{fancyhdr} % Cabeceras y pies de pagina
\usepackage{blindtext} % lorem ipsum dolor...
\graphicspath{ {images/} } % Ruta a las imagenes

\title{Práctica 1 - Técnicas de Búsqueda Local y Algoritmos Greedy para el Problema de la Mínima Dispersión Diferencial}
\author{Luis Miguel Guirado Bautista}
\date{10 de Abril de 2022 \qquad Curso 2021/2022}

\pagestyle{fancy}
\fancyhf{}

% do-while en algoritmos
\algdef{SE}[DOWHILE]{Do}{doWhile}{\algorithmicdo}[1]{\algorithmicwhile\ #1}%

\renewcommand*\contentsname{Índice} % Nombre del indice
\renewcommand{\figurename}{Figura} % Nombre de una figura
\renewcommand{\partname}{Ejercicio} % Nombre de \part
\renewcommand{\familydefault}{\sfdefault}

\begin{document}

    \begin{titlepage}
        \maketitle
        \thispagestyle{empty}
        \begin{enumerate}
            \item[\textbf{Correo:}]luismgb@correo.ugr.es
            \item[\textbf{DNI:}]75942712R
            \item[\textbf{Subgrupo:}]Martes, 3
            \item[\textbf{Problema:}]A (MDD) 
        \end{enumerate}
    \end{titlepage}

    \pagebreak

    \lhead{Práctica 1 - Metaheurísticas}
    \rhead{Luis Miguel Guirado Bautista}
    \tableofcontents

    \pagebreak
    \rfoot{\thepage}
    \section{Motivación}

    En esta práctica abordaremos el problema de la Mínima Dispersión Diferencial (MDD), que consiste en
    seleccionar un conjunto de $m$ elementos $M$ t.q $M \subset N$ y $m < n$, cuya dispersión entre sus
    elementos sea mínima, de modo que puede tratarse como un problema de optimización.

    \vspace*{0.1in}
    La dispersión de un elemento $v$ de la selección se interpreta como la suma de las distancias de
    $v$ al resto de puntos de $M$.

    \begin{equation*}
        \Delta(v) = \sum_{v \in M}d_{vu}
    \end{equation*}

    La dispersión de una solución $M$ se define como la diferencia entre los valores extremos de las
    dispersiones de los puntos de $M$.

    \begin{equation*}
        \Delta(M) = max_{v \in M}(u) - min_{u \in M}(v)
    \end{equation*}

    Siendo $\mathbb{M}$ el conjunto de posibles soluciones, podemos denotar la solución óptima $M^*$ como:

    \begin{equation*}
        M^* = min_{M \in \mathbb{M}}(\Delta(M))
    \end{equation*}

    Para solucionar de forma práctica este tipo de problemas usaremos 50 ficheros de datos generados por
    Glover, Kuo y Dhir en 1998 (GKD). Cada uno de estos ficheros son casos de problemas, con unos valores
    de $n$ y $m$ predefinidos, y la matriz diagonal superior de distancias euclídeas de cada uno de los
    puntos de $N$. Podemos recoger los datos de estos ficheros y representarlos de la siguiente manera.

    \begin{equation*}
        I_{nm} =
        \begin{pmatrix}
            0 & d_{01} & d_{02} & \dots & d_{0n} \\
            0 & 0 & d_{12} & \dots & d_{1_n} \\
            \vdots & \vdots & \vdots & \ddots & \vdots \\
            0 & 0 & 0 & \dots & d_{nn} \\
        \end{pmatrix}
    \end{equation*}

    \paragraph*{\textbf{Nota:}}Por simplicidad, el programa convertirá la matriz diagonal superior a una matriz diagonal superior e inferior.

    \vspace*{0.1in}
    También haremos uso de dos algoritmos implementados por el estudiante:
    \begin{itemize}
        \item Greedy
        \item Búsqueda Local
    \end{itemize}

    Finalmente haremos un análisis en base a los algoritmos empleados, el tamaño de los casos y el coste y tiempo medios.

    \pagebreak

    \section{Algoritmos}

    \subsection{Greedy}

    Consiste en seleccionar el siguiente elemento de la solución que minimice la dispersión con respecto a la
    solución actual. El primer elemento es aleatorio ya que $M = \emptyset$, y a partir de ahí se van escogiendo
    los elementos siguiendo el siguiente criterio.

    Para cada $u \notin M$:

    \begin{equation*}
        \partial (u) = \sum_{v \in M}d_{uv}
    \end{equation*}

    Y para cada $v \in M$, siendo $anterior(v)$ la distancia entre los puntos de $M$:

    \begin{equation*}
        \partial (v) = anterior(v)+d_{uv}
    \end{equation*}

    Entonces, siendo:

    \begin{equation*}
        \partial_{max}(u) = max(\partial (u), max_{v \in M}\partial (v))
    \end{equation*}

    \begin{equation*}
        \partial_{min}(u) = min(\partial (u), min_{v \in M}\partial (v))
    \end{equation*}

    La dispersión del siguiente punto $u$ será:

    \begin{equation*}
        g(u) = \partial_{max}(u) - \partial_{min}(u)
    \end{equation*}

    \begin{algorithmic}
        \Function{Greedy-MDD}{$distancias$, $N$, $m$}
            \State{$actual \gets \texttt{Aleatorio($N$)}$}
            \State{$s \gets [actual]$ \Comment{Lista de elementos pertenecientes a $N$}}
            \State{$disp \gets 0$}
            \While{$s.size < m$}
                \State{$(actual, disp) \gets \texttt{EscogerElemento($distancias$, $s$)}$} \Comment{Heurística empleada}
                \State{$s \gets s \cup actual$ \Comment{Se añade $actual$ a $s$}}
            \EndWhile
            \State{\Return{$(s$, $disp)$}}
        \EndFunction
    \end{algorithmic}
    \pagebreak
    \begin{algorithmic}
        \Function{EscogerElemento}{$distancias$, $s$}
        \State{$mejor \gets -$}
        \State{$min_g \gets \infty$}
        \For{$u \notin s$} \Comment{Para cada elemento no seleccionado}
            \State{$\partial(u) \gets \sum_{v \in s}distancias[u,v]$}
            \For{$v \in s$} \Comment{Para cada elemento seleccionado}
                \State{$\partial(v) \gets distancia(s) + distancias[u,v]$}
            \EndFor
        \State{$\partial_{max} \gets max(\partial (u), max_{v \in s}(\partial(v)))$}
        \State{$\partial_{min} \gets min(\partial (u), min_{v \in s}(\partial(v)))$}
        \If{$\partial_{max} - \partial_{min} < min_g$} \Comment{Nos quedamos con el de menor dispersión}
            \State{$mejor \gets u$}
            \State{$min_g \gets \partial_{max} - \partial_{min}$}
        \EndIf
        \EndFor
        \State{\Return{$mejor$, $min\_g$}}
        \EndFunction
    \end{algorithmic}

    \subsection{Búsqueda Local}

    Dada una solución completamente aleatoria $M$, consiste en ir generando soluciones vecinas que mejoren
    el resultado de la solución original $M$ dentro de un entorno $E(M)$, hasta que, generado todo el entorno de
    una solución, no se haya encontrado una mejora o se haya alcanzado un número máximo de iteraciones definida por
    el programador (en este caso, el máximo de iteraciones será de $10^5$).
    
    Este algoritmo requiere definir el
    $E(M)$ y un operador que genere una solución vecina a $M$ dentro de su entorno.
    Para ello, definiremos el método \texttt{Int(M,i,j)}, que se encargará de intercambiar un elemento $i \in M$ por
    otro $j \notin M$ para que quede una solución vecina $M'$ t.q. $i \notin M'$ y $j \in M'$.

    Entonces es fácil deducir que el entorno de $M$ son todas las soluciones vecinas $M'$ que puede generar
    el operador \texttt{Int(M,i,j)}, que son $m \cdot (n-m)$ vecinos. No obstante, el entorno puede llegar a ser muy grande,
    así que realizaremos una factorización del coste para cada vecino que consideremos para aumentar la eficiencia.

    El coste de realizar un intercambio es:
    
    \begin{equation*}
        Z_{mm}(M,i,j) = (\partial_{max} - \partial_{min}) - Z_{mm}(M) = Z_{mm}(M') - Z_{mm}(M)
    \end{equation*}

    \begin{equation*}
        \partial_{max} = max(\partial (v), max_{w \in M}(\partial (w)))
    \end{equation*}

    \begin{equation*}
        \partial_{min} = min(\partial (v), min_{w \in M}(\partial (w)))
    \end{equation*}

    Para todo elemento $v$:
    \begin{equation*}
        \partial (v) = \sum_{w \in M}d_{vw}
    \end{equation*}

    Luego para todo elemento $w \in M$, siendo $anterior(w)$ el coste de $M$:
    \begin{equation*}
        \partial (w) = anterior(w) - d_{wu} + d_{wv}
    \end{equation*}

    El algoritmo de búsqueda local escogerá al vecino como buena solución si $Z_{mm}(M,i,j) < 0$, es decir, si al
    realizar el intercambio, se ha producido una mejora en cuanto al coste con respecto a la solución original.

    \pagebreak

    \begin{algorithmic}
        \Function{BúsquedaLocal}{$distancias$, $N$, $m$, $max\_iters$}
            \State{$actual \gets \texttt{GenerarSolucion($N$,$m$)}$}
            \State{$iters \gets 0$}
            \Do
                \State{$iters \gets iters + 1$}
                \State{$prima \gets \texttt{EscogerVecino($distancias$, $actual$, $N$)}$}
                \If{$\texttt{Distancia($prima$)} < \texttt{Distancia($actual$)}$}
                    \State{$actual \gets prima$}
                \EndIf
            \doWhile{$iters \geq max\_iters$ \textbf{or} $\texttt{Distancia($prima$)} \geq \texttt{Distancia($actual$)}$}
            \State{\Return{$actual$, $\texttt{Distancia($actual$)}$}}
        \EndFunction
    \end{algorithmic}
    \vspace*{0.2in}
    \begin{algorithmic}
        \Function{EscogerVecino}{$distancias$, $actual$, $N$}
            \For{$u \in actual$}
                \For{$v \notin actual$}
                    \State{$M' \gets \texttt{Int($actual, u, v$)}$} \Comment{Operador de intercambio}
                    \State{$\partial (v) \gets \sum_{w \in M'}d_{vw}$}
                    \For{$w \in M'$}
                        \State{$\partial (w) \gets \texttt{Distancia($actual$)} - d_{wu} + d_{wv}$}
                    \EndFor
                    \State{$\partial_{max} \gets max(\partial (v), max_{w \in M'}( \partial (w)))$}
                    \State{$\partial_{min} \gets min(\partial (v), min_{w \in M'}( \partial (w)))$}
                    \State{$Z_{mm}(M') \gets \partial_{max} - \partial_{min}$}
                    \State{$Z_{mm}(M,u,v) \gets Z_{mm}(M') - Z_{mm}(M)$}
                    \If{$Z_{mm}(M,u,v) < 0$}
                        \State{\Return{$M', Z_{mm}(M')$}}
                    \EndIf
                \EndFor
            \EndFor
            \State{\Return{$M'$}}
        \EndFunction
    \end{algorithmic}
    \vspace*{0.2in}

    \pagebreak

    \section{Desarrollo}

    La práctica se ha realizado en Python 3.10.2, haciendo uso de los paquetes \texttt{time},
    \texttt{NumPy}, \texttt{os}, \texttt{Pandas} y \texttt{random}. No se ha utilizado ningún framework.
    \begin{enumerate}
        \item[]Para la generación de la semilla se ha usado la distribución uniforme discreta de \texttt{NumPy}
        a partir del método \texttt{randint}.
        \item[]Para la lectura de los ficheros de datos GKD se ha usado el método \texttt{listdir} de \texttt{os}.
        y para su organización se han usado las estructuras de datos \texttt{array} de \texttt{NumPy}.
        \item[]Para la medición de tiempos se ha usado el paquete \texttt{time}.
        \item[]Para la gestión de datos en los algoritmos, además de las estructuras de datos primitivas de 
        Python como \texttt{list}, \texttt{tuple}, o \texttt{dict}, se ha usado la estructura de datos \texttt{array}.
        de \texttt{NumPy} y varios de sus métodos como \texttt{sum, zeros, shape (atributo)},...
        \item[]Para la generación de las tablas de las que se hablará posteriormente se ha usado el paquete \texttt{Pandas}.
    \end{enumerate}

    \section{Análisis}

    Una instancia de ejecución de todo el experimento (5 ejecuciones de cada caso con cada uno de los algoritmos),
    se encuentra en el archivo \texttt{Tablas\_MDD\_2021-22.ods}. No obstante, al ejecutar el programa, siempre genera
    una tabla del mismo formato (salvo tablas finales, no son válidas). Basta con ejecutar el archivo mediante cualquier
    intérprete, aunque es necesario tener los archivos GKD en el directorio \texttt{/datos/p1a/} y tener instalados los paquetes mencionados
    en la sección de desarrollo.

    \paragraph*{\textbf{Conclusión en base a los resultados}}
    \emph{
        Aquí seré algo informal. Esta práctica ha sido un desastre, me ha costado muchas veces más entender las
        heurísticas que todo lo demás junto (implementación del algoritmo general, realización del informe y tablas, etc.)
        y no ha sido por falta de tiempo, sino por que "no daba en el clavo". De ahí, los resultados que comentaré a continuación.
        Aunque he estado observando los resultados y los tiempos bajan en Greedy según n, pero sobretodo por m, ya que escoge los elementos de
        forma secuencial, y cuanto más elementos tenga que analizar y escoger, obviamente, más tardará. Aunque BL tiene un error de
        implementación que se basa en el tiempo, los tiempos son más uniformes y tardan muchísimo menos que en Greedy.
        Y el error más grave que he tenido es la desviación media en BL, siendo mayor que $2\cdot 10^5$.
    }

\end{document}